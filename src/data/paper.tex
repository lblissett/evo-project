\documentclass[12pt,
    a4paper,
    headinclude,
    footinclude]{scrreprt}

    %plainfootsepline

\usepackage{blindtext}
\usepackage[utf8]{inputenc}
\usepackage{setspace}
\usepackage[ngerman]{babel}
\usepackage[ngerman, num]{isodate}
\usepackage[left=3cm,right=2cm,top=2.8cm,bottom=1.6cm]{geometry}

\usepackage[style=numeric]{biblatex}
\usepackage[babel,german=guillemets]{csquotes}

\usepackage{float}

\usepackage{graphicx}
\usepackage{tabularx}
\usepackage{caption}

\usepackage{listings}
\usepackage{color}
\renewcommand{\lstlistingname}{Auflistung}
\renewcommand{\lstlistlistingname}{Auflistungsverzeichnis}
\definecolor{hellgrau}{rgb}{.95,.95,.95}
\definecolor{grau}{rgb}{.9,.9,.9}
\lstset{
	%backgroundcolor=\color{hellgrau},
	%backgroundcolor=\color{blue}
	%basicstyle=\scriptsize\ttfamily,
	%keywordstyle=\bfseries\ttfamily\color{orange},
	%stringstyle=\color{green}\ttfamily,
	%commentstyle=\color{middlegray}\ttfamily,
	%emph={square}, 
	%emphstyle=\color{blue}\texttt,
	%emph={[2]root,base},
	%emphstyle={[2]\color{yac}\texttt},
	showstringspaces=false,
	%flexiblecolumns=false,
	tabsize=1,
	numbers=left,
	%numberstyle=\tiny,
	numberblanklines=false,
	stepnumber=1,
	%numbersep=0pt,
	xleftmargin=20pt,
	frame=l,
	framesep=4.5mm,
	framexleftmargin=2.5mm,
	fillcolor=\color{grau},
	rulecolor=\color{grau},
	%numberstyle=\normalfont\tiny\color{numbercolor}
}

\usepackage{wrapfig}

\bibliography{res33.bib} 


\AtBeginDocument{\setlength{\glslistdottedwidth}{.15\columnwidth}}
\usepackage[acronyms,style=listdotted,shortcuts,translate=babel,toc]{glossaries}
%\GlsSetXdyLanguage{german}                         % Deutsche Spracheinstellung (nicht ngerman!) 
%\GlsSetXdyCodePage{duden-utf8}                     % Deutsche Codierung 
%\newglossarystyle{glosstil}{                     % neuer Stil mit Namen
%	\glossarystyle{listdotted}                           % basierent auf Stil long
%	\renewenvironment{theglossary}
%	{\begin{longtable}
%			{@{}p{0.1\textwidth} p{0.8\textwidth}}}   % einstellen der Spaltenbreiten
%		{\end{longtable}}
%	\renewcommand*{\glsgroupskip}{}               % keine Gruppenumbrüche
%} 

%\renewcommand*{\glsgroupskip}{}
%\makeglossaries
% \newacronym{TCP}{TCP}{Transmission Control Protocol}
% \newacronym{FTP}{FTP}{File Transfer Protocol}
% \newacronym{IP}{IP}{Internet Protocol}
% \newacronym{DNS}{DNS}{Domain Name Server}
% \newacronym{WWW}{WWW}{World Wide Web}
% \newacronym{HTTP}{HTTP}{Hypertex Transfer Protocol}
% \newacronym{HTTPS}{HTTPS}{Hypertex Transfer Protocol Secure}
% \newacronym{HTML}{HTML}{Hypertex Markup Language}
% \newacronym{XML}{XML}{Extensible Markup Language}
% \newacronym{WS}{WS}{WebSocket}
% \newacronym{ASCII}{ASCII}{American Standard Code for Information Interchange}
% \newacronym{UTF-8}{UTF-8}{8-Bit Universal Character Set Transformation Format}
%\usepackage{etoolbox}

%\newacronym{}{}{}

%\usepackage{fontspec}
%\setmainfont{Times New Roman}
%\usepackage[T1]{fontenc}
%\newcommand{\changefont}[3]{
%\fontfamily{#1} \fontseries{#2} \fontshape{#3} \selectfont}

%\renewcommand{\chapterpagestyle}{scrheadings}


%kopf und fusszeile
%\usepackage[headsepline]{scrpage2}
%\pagestyle{scrheadings} 
%\setlength{\footskip}{8mm}
%\clearscrheadings
%\ihead{\leftmark}
%\ohead{\rightmark}
%\automark[section]{chapter}
%\cfoot{\pagemark}
%kopf und fusszeile

%Schriftart sfffamily serifenlos
\setkomafont{pageheadfoot}{\normalfont\rmfamily\bfseries}

\setkomafont{chapterentry}{\normalfont\rmfamily\bfseries}
%\setkomafont{sectionentry}{\normalfont\rmfamily}

\setkomafont{chapter}{\huge\normalfont\rmfamily\bfseries}
\setkomafont{section}{\Large\normalfont\rmfamily\bfseries}
%Schriftart



\DefineBibliographyStrings{ngerman}{%
	bibliography={Literaturverzeichnis}% NICHT references
}


\author{Martin Braun}
\title{Redfish}

\begin{document}
	\onehalfspacing
	\monthyearsepgerman{\,}{\,}
	\setcounter{tocdepth}{2}
	

	
	\begin{titlepage}
	
		\begin{center}
			~\\[2cm]
			Berufsakademie Sachsen \\
			Staatliche Studienakadamie Leipzig \\[2.4cm]
           
			\begin{huge}
			    \textbf{Binäre und reelle Kodierung im Vergleich} \\[2.4cm]
			\end{huge}
			
			\doublespacing

			Evolutionäre Algorithmen 


		\end{center}
		
		\onehalfspacing
		\begin{tabbing}
			Eingereicht von: \= ~ \= ~ \= ~ \= Georg Andrassy \\
			\> \> \> \> Martin Braun \\
			\\

		\end{tabbing}
		\vspace*{\fill}
		Leipzig, \today
		
	\end{titlepage}
    
    %Inhaltsverzeichnis
    \pagenumbering{Roman}
%    \tableofcontents 	
    \clearpage
    %Inhaltsverzeichnis
        
    \pagenumbering{arabic}
    \setcounter{page}{2}
    
\section*{1. Einführung}	\onehalfspacing

Zur näherungsweisen Berechnung des globalen Minimums der Griewank-Funktion: \[f(x) = 1 +  \sum_{i=1}^n \frac{x_i^{2}}{400n} -  \prod \limits_{i=1}^n cos \left(\frac{x_i}{\sqrt{i}}\right)\] wird ein evolutionärer Algorithmus eingesetzt. Dieser Algorithmus wird vor allem im Hinblick auf den Einfluss  a) unterschiedlicher Kodierungsarten b) stabiler und wachsender Populationsgrößen auf den Fitnesswert der Individuen untersucht. Zusätzlich wird die Differenz zwischen dem besten und dem schlechtesten Individuum in der Population als Maß der Streuung betrachtet. Die Minimierung der Griewank-Funktion wird für drei verschiedene n (n=5, 20, 50) im Wertebereich von -512 bis 511 getestet:




\section*{2. Umsetzung}


Für die Umsetzung wird folgender Algorithmus verwendet:

\begin{enumerate}
	\item Erzeugung der Startpopulation 
	\begin{itemize}
		\item fixe Ausgangspopulation mit X Anfangsindividuen
	\end{itemize}
	\item Elternselektion
		\begin{itemize}
		\item Anzahl Elternpaar = e mit zufälliger Selektion
		\item e wachsend bzw. stabil
	\end{itemize}
	\item Rekombination
		\begin{itemize}
		\item Reelle Kodierung: intermediäre Rekombination
		\item Binärkodierung: Ein-Punkt- und Zwei-Punkt-Rekombination (Zwei-Punkt: Mittelteil austauschen)
	\end{itemize}
	\item Mutation
		\begin{itemize}
		\item Mutationswahrscheinlichkeit M [0,1]
		\item von jedem Individuum soll zufällig ein Gen ausgewählt werden
		\item für Mutation eines Gens Zufallszahl z [0,1] ermitteln
		\item Reelle Kodierung: wenn z $<$ M, dann mutiere: addiere zum gewählten Allel einen festen Wert W
		\item Binärkodierung: zufällig von dem gewählten Binär-Allel ein Bit auswählen, wenn z $<$ M, dann switche Bit von 0 zu 1 bzw. 1 zu 0
	\end{itemize} 
	\item Umweltselektion
		\begin{itemize}
		\item aus Eltern + Kindern soll neue Population gewählt werden 
		\item zuerst die besten A Individuen nehmen (deterministische Selektion) Gesamtpopulation soll wachsen (A wächst linear von Generation zu Generation um +1 an)
		\item dann aus Rest zufällig B Individuen auswählen
	\end{itemize}
	\item gehe zu Punkt 2 bis Abbruchkriterium K erreicht
\end{enumerate}

\begin{table}[h]
	\centering
	\caption*{Verwendete Parameter}
	\begin{tabularx}{14cm}{|p{9cm}|X|}
		\hline
		$Parameter$ & $Wert$ \\
		\hline
		\hline
		Größe der Startpopulation X Individuen& 10  \\
				\hline
		Anzahl der Elternpaare e & 10 (+1 pro Zyklus)  \\
				\hline
		Rekombinationswahrscheinlichkeit& 0.9 \\
				\hline
		Mutationswahrscheinlichkeit M & 0.1 \\
				\hline
		reeller Mutationswert W& +5.0  \\
				\hline
		Auswahl bester Individuen a& 10 (+1 pro Zyklus) \\
		\hline
		Auswahl zufällige Umweltselektion B& 3 \\
		\hline
		Abbruchkriterium K Zyklen & 1500\\
		\hline
	\end{tabularx}
\end{table}


	
\section*{3. Ergebnisse und Interpretation}	

Beim Vergleich der Kodierungsarten fällt auf, dass die beiden binären Kodierungen sich schneller zum globalen Minimum orientieren und bessere Fitnesswerte bringen. Die reele Kodierung liefert schlechtere Ergebnisse.
		\begin{figure}[H]
			\includegraphics[width=0.75\textwidth]{best-20-stable.jpeg}
			\caption*{Kodierung im Vergleich bei n=20 (stabile Population)} 
			\label{InputOutput}
		\end{figure}

Dieses Ergebnis wird auch bei einer wachsenden Bevölkerung erreicht. Im Unterschied zur stabilen Bevölkerung liefert der Algorithmus aber bessere Ergebnisse. D.h. es werden schneller bessere Fitnesswerte erreicht und die binäre Kodierungsarten finden mit $<$ 200 Zyklen und n=20 das globale Minimum von 0.
	\begin{figure}[H]
	\includegraphics[width=0.75\textwidth]{best-20-grow.jpeg} 
	
	\caption*{Kodierungen im Vergleich bei n=20 (wachsende Population)} 
	\label{InputOutput}
\end{figure}


Bei Betrachtung der Streuung der Individuen (d.h. dem Abstand des besten zum schlechtesten Individuum) ist auffällig, dass sich die Individuen bei der reelen Kodierung sehr schnell sehr ähnlich werden. Die Individuen bei der binären Kodierung streuen etwas mehr.
		\begin{figure}[H]
		\includegraphics[width=0.75\textwidth]{best-and-worst.jpeg} 
		
		\caption*{Streuung der Individuen} 
		\label{InputOutput}
	\end{figure}


	
	

	

	
	

	

	


	

	
	
	



%	\glsnogroupskiptrue
%	\printglossary[type=\acronymtype, title=Abkürzungsverzeichnis]
	
%	\listoffigures
%	\addcontentsline{toc}{chapter}{Abbildungsverzeichnis}
	
%	\lstlistoflistings
%	\addcontentsline{toc}{chapter}{Auflistungsverzeichnis}
	
%	\printbibliography 
%	\addcontentsline{toc}{chapter}{Literaturverzeichnis}
	
	
	%\begin{acronym}
	%	\acro{TCP}{Transmission Control Protocol}
	%	\acro{FTP}{File Transfer Protocol}
	%	\acro{IP}{Internet Protocol}
	%	\acro{DNS}{Domain Name Server}
	%	\acro{WWW}{World Wide Web}
%	\end{acronym}

\end{document}